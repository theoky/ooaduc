%% Example for the 
%% Object-Oriented Analysis and Design (OOAD): Use Cases (UC)
%% Style File
%%---------------------------------------------------------------------------
%%
%% T. Kopetzky
%% Date: 20/10/2014
%%
% This work may be distributed and/or modified under the conditions of the 
% GNU General Public License version 3 GPLv3
% (see http://www.gnu.org/licenses/gpl-3.0.html).
\begin{usecase}{Disarm Alarm System} 

\scope{Authentication and system disabling}
\level{Primary Task}

\goal{Quick, safe and straight forward disarming of the alarm system}

\primaryactor{House Owner}
\secondaryactor{Police}

\preconditions{
\item Alarm system is armed and active (see UC\ref{uc:arm} on page~\pageref{uc:arm}).
\item User knows the disarming procedure and remembers the password.
}

\successfuloutcome{System is disarmed.}
\failureoutcomes{
	\failureoutcome{Password has been entered too slowly.}
	{System resp. alarm goes off.}
}

\mainsuccess{
\item User enters house.
\item System starts timer for alarm to go off.
\item User enters 4 digit password.\label{uc2:return1}
\item System recognizes password and disarms alarm.\label{uc2:alt1}
\item Use case ends successfully.
}

\extensions{
	\item[\ref{uc2:alt1}.a] Password wrong:
		\begin{enumerate*}
		\item System recognizes wrong password and signals wrong password.
		\item Use case continues with step~\ref{uc2:return1}
		\end{enumerate*}
	\item[\ref{uc2:alt1}.b] Timer goes off.
		\begin{enumerate*}
		\item System recognizes that time to enter password is over and sets off the alarm.
		\item Use case ends with failure.
		\end{enumerate*}
}

\miscellaneous{
	\item It takes user approx. 1 minute to reach kitchen after entering through front door and taking off shoes and coat.
	\item Definition:
	Alarm goes off = sound siren at the house and notify police
}

\openissues{
\issue{Should the alarm go off immediately if house is entered through back door?}
}
\end{usecase}
